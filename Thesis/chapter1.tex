% This file contains CHAPTER ONE

\chapter{Introduction}


\section{Motivation}
The divisor class group of a hyperelliptic curve defined over a finite field is
a finite abelian group at the center of a number of important open questions in
algebraic geometry and number theory. Sutherland~\cite{Sutherland_g3_2019}
surveys some of these, including the computation of the associated $L$-functions
and zeta functions used in his investigation of Sato-Tate
distributions~\cite{Sutherland_sato_2016}. Many of these problems lend
themselves to empirical investigation, and as emphasized by Sutherland, fast
arithmetic in the divisor class group is crucial for their efficiency. Indeed,
implementations of these fundamental operations are at the core of the algebraic
geometry packages of widely-used computer algebra systems such as Magma and
Sage.

All hyperelliptic curves are represented as models that are categorized as
either ramified (imaginary), split (real), or inert according to their number of
points at infinity defined over the base field. Ramified curves have one point
at infinity, whereas split curves have two. Inert (also called unusual) curves
have no infinite points defined over the base field and are usually avoided in
practice as they have cumbersome divisor class group arithmetic and can be
transformed to a split model over a quadratic extension of the base field.

Divisor class group arithmetic differs on ramified and split models. Many
efficient algorithms have been proposed for the ramified setting, notably due to
its extensive use in cryptographic applications. The split scenario is more
complicated. As a result, optimizing divisor arithmetic on split hyperelliptic
curves has received less attention from the research community. However, split
models have many interesting properties; most importantly, they exist for a
large array of hyperelliptic curves that cannot be described with a ramified
model. Thus, exhaustive computations such as those
in~\cite{Sutherland_sato_2016} require working on split models by necessity.

Arithmetic in the divisor class group of a hyperelliptic curve can be described
algebraically using an algorithm due to Cantor~\cite{Cantor_compjac_1987}, and
expressed in terms of polynomial arithmetic. Various improvements and extensions
to Cantor's algorithm have been proposed for ramified model curves, including an
adaptation of Shank's NUCOMP algorithm~\cite{ShanksNUCOMP} for composing binary
quadratic forms~\cite{jacobson_nucomp_2002}. The main idea behind NUCOMP is that
instead of composing two divisors directly and then reducing to find an
equivalent reduced divisor, a type of reduction is applied part way through the
composition, so that when the composition is finished the result is almost
always reduced.  The effect is that the sizes of the intermediate operands are
reduced and intermediate divisor class representations need not be computed,
resulting in better performance in most cases. Improvements to NUCOMP have been
proposed, most recently the work of~\cite{ImbJac13:amc}, where best practices
for computing Cantor's algorithm and NUCOMP are empirically investigated. 

NUCOMP has also been proposed for arithmetic in the so-called infrastructure of
a split model curve~\cite{jacobson_fast_2007}. However, as shown by Galbraith
et. al.~\cite{Galbraith_balanced_2008,Morales_balanced_2009}, arithmetic on
split model hyperelliptic curves is most efficiently realized via a divisor
arithmetic framework referred to as balanced. Although the balanced and the
infrastructure frameworks are similar, NUCOMP had yet to be applied explicitly
to the former.

Several optimized algorithms, known as explicit formulas, are available for
divisor class arithmetic in the divisor class group of genus 2 and 3
hyperelliptic curves. These algorithms, which describe divisor class arithmetic
in terms of as few field operations as possible, are typically highly
specialized for certain families of curves with efficient computational
properties. This is too restrictive, as curves used for number theoretic
applications are generic and cannot be chosen with computationally efficient
properties. These algorithms also only compute the most common divisor input
cases under the assumption that the arithmetic is computed over
cryptographically large finite fields of size
128-bits~\cite{BosCostelloHisilLauter_fastg2_2013}, and the uncommon cases are
instead computed using polynomial arithmetic. Furthermore, field additions are
often not even considered. This is not ideal for number theoretic applications,
as the finite field sizes are relatively much smaller.



\section{Summary of Research Contributions}
This thesis provides contributions to improve the efficiency of divisor class
arithmetic on hyperelliptic curves, with special attention to the often ignored
split model cases and to genus 2 and 3. There are two main contributions; the
introduction of ``Balanced NUCOMP'' for improved arithmetic on curves given by
split models of arbitrary genus, and improved explicit formulas for genus 2
(ramified and split models) and genus 3 (split models). Empirical analysis is
conducted for all contributed algorithms to provide proof of correctness and
comparisons to previous best. Details for both main contributions are described
next.




\subsection{Balanced NUCOMP}
The first contribution of this work is a generalization of NUCOMP for divisor
class arithmetic to the  balanced setting of a divisor class group over split
model hyperelliptic curves. All best practices from previous works in the
ramified model setting are incorporated and new balanced setting-specific
improvements that further enhance practical performance are introduced.
Specifically, this novel version of NUCOMP includes various improvements over
its infrastructure counterpart~\cite{jacobson_fast_2007}:
\begin{itemize}
    \item Describes for the first time exactly how to use NUCOMP in the
    framework of balanced divisors, including explicit computations of the
    required balancing coefficients;
    \item Introduces a novel normalization of divisors in order to eliminate
    the extra adjustment step required in \cite{jacobson_fast_2007} for typical
    inputs when the genus of the hyperelliptic curve is odd, so that in all
    cases typical inputs require no extra reduction nor adjustment steps;
    \item Uses certain aspects of NUCOMP to compute one adjustment step
    almost for free in some cases.
\end{itemize}

In part, the efficiency of Balanced NUCOMP is achieved via a new normalization
of the Mumford $v$ polynomial dubbed "negative reduced basis", that eliminates
the need for extra adjustment steps over odd genus split model curves introduced
in prior work on arithmetic in the infrastructure of split model
curves~\cite{jacobson_fast_2007}. The use of the negative reduced basis does not
affect the efficiency of arithmetic over even genus split model curves, and
closes the gap between the two cases discussed in~\cite{jacobson_fast_2007}. The
new basis takes advantage of the  inequality between the number of the two types
of adjustments required in divisor class addition, that is inherently part of
the definition of a balanced divisor class. Over even genus split model curves,
the inequality does not exist, and the change in normalization to $v$ has no
effect.

The development of Balanced NUCOMP also involved incorporating and accounting
for the balancing coefficient $n$ from the balanced setting over split model
curves. A proof of correctness is provided, relating Cantor's algorithm applied
to balanced divisor arithmetic to Balanced NUCOMP. In contrast to the NUCOMP
algorithm proposed in~\cite{jacobson_fast_2007}, this work applies NUCOMP
techniques to special low input degree divisor classes in order to combine
addition and adjustments steps together using NUCOMP techniques.

Finally, implementation of the work in Magma was developed as proof of concept
and for empirical timing purposes. Thorough empirical timing analysis comparing
the different reduced bases considered, as well as comparing Balanced NUCOMP to
previous best, are provided as support for the algorithm design choices made.
The analysis demonstrates the efficiency gains realized from the novel version
of NUCOMP as compared with the previous best balanced divisor class group
arithmetic based on Cantor's algorithm and the arithmetic implemented in Magma,
showing that NUCOMP is the method of choice for all but the smallest genera. In
such cases overhead introduced by NUCOMP negates the benefits, where the
cross-over point is implementation dependent.

With these improvements, NUCOMP is more efficient than Cantor's algorithm for
genus as low as 5, compared to 7 using the version in \cite{jacobson_fast_2007}.
The implementation developed in this work is faster than Magma's built-in
arithmetic for $g \geq 7$, and the gap increases with the genus. Magma is closed
source software and therefore an analysis of Magma's implementation was not
possible. Magma does document that built-in arithmetic takes advantage of C
implementations, and in comparison any implementations written in Magma require
passing through Magma's interface, consequently incurring additional
computational overhead in comparison.



\subsection{Explicit Formulas for Genus 2 and 3 Arithmetic}
The second contribution is novel explicit formulas for divisor addition and
doubling on generic hyperelliptic curves in Weierstrass form for both ramified
and split models of genus 2 and split models of genus 3. These formulas include
several contributions:
\begin{itemize}

    \item New streamlined explicit formulas for computing divisor class
    arithmetic over genus 2 and 3 split model curves using the balanced
    framework are developed based on the novel Balanced NUCOMP algorithm
    introduced in this thesis. The split model formulas in this work are the
    first to completely encapsulate divisor class arithmetic in the balanced
    framework, whereas previous split model formulas were developed for
    computing arithmetic in the
    infrastructure~\cite{EricksonJacobsonStein_realg2_2011, rad2019jacobian}.
    \item All previous literature on computing genus 2 and 3 divisor doubling
    and addition was surveyed, and through combining the best aspects of
    previous works and novel techniques introduced here, the number of field
    operations in almost all cases over genus 2 ramified and genus 2 and 3 split
    curves were reduced. The reduction in field additions is especially
    significant. The relative cost of a field addition is non trivial when
    compared to a field multiplication; for example three additions are
    considered equivalent in cost to one multiplication in the work of
    Sutherland~\cite{Sutherland_g3_2019}. Ramified model genus 3 formulas were
    omitted as these are work in progress by a student supervised by the author of this work. 
    \item All genus 2 ramified and genus 2 and 3 split model formulas developed
    are complete in the sense that all possible cases are explicitly computed,
    and every case requires exactly one inversion. Previous
    work~\cite{EricksonJacobsonStein_realg2_2011, rad2019jacobian,
    Sutherland_g3_2019} omitted explicit computation of most non-typical input
    cases. 
\end{itemize}

First, the general approaches on which explicit formulas are based in previous
work were considered  and a complete survey of all prior explicit techniques was
conducted during the development process. NUCOMP and Balanced NUCOMP were
considered as possibilities, although these were not part of any previous work
on explicit formulas.

NUCOMP and Balanced NUCOMP based approaches specialized to genus 2 and Balanced
NUCOMP specialized to genus 3 for ramified and split model curves produced the
simplest formulas requiring fewer operation counts than any prior work. NUCOMP
based approaches efficiently describe special cases computations; taking
advantage of this resulted in explicit formulas that are complete, and require
only one inversion for any computation path. The split model formulas in this
thesis are the first to completely encapsulate divisor class arithmetic in the
balanced setting, whereas previous split model formulas were developed for
computing arithmetic in the
infrastructure~\cite{EricksonJacobsonStein_realg2_2011, rad2019jacobian} or
considered only the most frequently occurring divisor class
input~\cite{Sutherland_g3_2019} cases.

Implementations of all formulas were developed in Magma as proof of concept.
Explicit formulas are very complex and prone to errors. Both black-box and
white-box testing programs were utilized to provide thorough evidence that the
formulas as they appear in the Magma code are all correct. Black-box testing
computes numerous additions of randomly chosen divisor classes over randomly
chosen curve models and compares the output of each addition to that obtained
using Cantor's Algorithm. White-box testing uses specially-selected divisor
class additions that are guaranteed to use every possible computation path
within each formula, and similarly compares the output to Cantor's Algorithm.
Moreover, empirical timing tests were developed in Magma, including
implementations of previous work, to compare the work of this thesis with prior
work on the same platform.

Following~\cite{Sutherland_g3_2019}, the formulas use an affine model where each
operation requires one field inversion. In the context of computational number
theoretic applications like those in~\cite{Kedlaya_lseries_2008}, affine
formulas are superior to the inversion-free formulas obtained using projective
coordinates because group order algorithms such as baby-step giant-step require
frequent equality tests. As representations of group elements using projective
coordinates are not unique, their use would incur a non-negligible computational
cost per equality test, and precludes the use of more efficient searchable data
structures for the baby steps such as hash tables. Furthermore, as described in
~\cite{Sutherland_g3_2019}, the inversions can often be combined using a trick
due to Montgomery, allowing a batch of inversions to be computed with only one
field inversion and a small number of field multiplications. Thus, in this work,
as cryptographic applications are not our motivation, only affine formulas are
developed. If projective formulas are required for some other application, our
formulas can readily be converted to that setting using standard practices.

Although the applications mentioned above involve hyperelliptic curves defined
over finite fields, versions of our formulas that work over any field, as well
as versions specialized to fields of characteristic not equal to two and fields
of characteristic two are included. For the latter two cases, the  formulas take
advantage of simplifications that result from canonical forms of the
hyperelliptic curve equation in which several terms are zero. Magma
implementations are given for all of the formulas in this thesis and are
available at \url{https://github.com/salindne/divisorArithmetic}. 


\section{Organization of Thesis}
In Chapter~\ref{cha:bg}, the mathematical background of hyperelliptic curves
limited to the material essential to understanding this thesis is introduced.
Basic definitions and properties of hyperelliptic curves, divisors, the divisor
class group and the group law arithmetic over the divisor class group are
provided. The most efficient divisor class group setting dubbed ``balanced
setting" is introduced for split model curves, using balanced representations of
divisor classes. Most definitions, theorems and algorithms are followed by
examples for better understanding.

The purpose of Chapter~\ref{cha:nucomp} is the introduction of a novel
adaptation of NUCOMP to the balanced setting of the divisor class group over
split model curves. A general algorithm called Balanced NUCOMP for divisor class
arithmetic over all genus is introduced, with discussion of the development
process and algorithm design choices. Empirical analysis producing evidence for
the improvements Balanced NUCOMP brings over previous best. The chapter begins
with a variety of improvements to generic genus polynomial-based divisor class
addition algorithms from previous works, along with background on the NUCOMP
algorithm over ramified model curves.

Chapter~\ref{cha:towardsExplicit} introduces generic approaches for, basic
definition of, and techniques for developing explicit formulas from previous works.
A novel approach based on NUCOMP for ramified curves, and Balanced NUCOMP for
split curves is also described for genus 2 and 3 curves, where the development of all
explicit formulas, including special cases, is unified for each setting
respectively. All novel explicit formula techniques introduced in
this thesis are provided at the end of the chapter.

Chapters~\ref{cha:g2} and~\ref{cha:g3} introduce the novel formulas
developed in this thesis for genus 2 and 3 respectively. Chapter~\ref{cha:g2} is
split into major sections, genus 2 ramified model formulas in
Section~\ref{sec:g2ramExpl}, and genus 2 split model formulas in
Section~\ref{sec:g2splitExpl}. Both sections and Chapter~\ref{cha:g3} have
similar structure where first curve simplifications via isomorphisms are
introduced and prior work in each setting is discussed. The basic formulations
of doubling and addition algorithms required for complete doubling and addition
are then presented, followed by a discussion and comparison of the field
operations costs compared to literature. Both Chapters~\ref{cha:g2}
and~\ref{cha:g3} end with an empirical timing analysis of the formulas compared to
previous best computed in Magma using the same platform. The empirical results
provide evidence supporting the assertions made in the operation cost
comparisons.

A summary of the work done in this thesis and suggestions for potential research
for the future are given in Chapter~\ref{cha:conclusion}. All novel
implementations were solely developed by the author of this thesis. Some
previous best explicit formulas implemented in Magma were available from the
respective authors of previous literature, all others were implemented here,
based on the respective papers.

