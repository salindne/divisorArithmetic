% This file contains CHAPTER SEVEN

\chapter{Conclusion} 
\label{cha:conclusion}
In this thesis, efficient computational approaches for computing divisor class
arithmetic on ramified and split model curves were developed, introducing
several novel algorithms that outperform previous best. Balanced NUCOMP for
generic genus split model curves, explicit formulas based on NUCOMP and Balanced
NUCOMP for genus 2 ramified and split model curves, and genus 3 split model
curves were developed, and thoroughly tested. Balanced NUCOMP introduces a new
"negative reduced" basis for the normalization of the $v$ Mumford polynomial,
eliminating the inefficiencies for odd genus split model curves found in
previous work on arithmetic in the infrastructure of a split model curve. Moreover,
in previous work it was suggested that NUCOMP does not benefit divisor class
arithmetic over hyperelliptic curves of small genera, but explicit formulas
based on NUCOMP over genus 2 ramified model curves and Balance NUCOMP over genus
2 and 3 split model curves prove to be computationally more efficient than
previous best in the above cases mentioned. Most notably over split model genus 3
curves, there is a considerable improvement overall in the operation counts of
divisor class addition and doubling supported by empirical evidence using Magma.
Ramified model genus 3 arithmetic would also benefit from NUCOMP techniques but
was not considered in this thesis due to ongoing work elsewhere; this will be
presented along with our split model results in a forthcoming paper. All
algorithms are implemented, tested for correctness, and empirically tested for
comparisons to previous best. All software developed in support of this thesis
is available at \url{https://github.com/salindne/divisorArithmetic}.


\section{Future Work}
\label{sec:futurework}
The work of this thesis has to potential to be extended in the following ways:

Adapt NUCOMP to divisor arithmetic over non-hyperelliptic $C_{a,b}$ curves as
this setting also plays a role in computational number theoretic
applications~\cite{Sutherland_sato_2016}. Algorithms for divisor arithmetic on
generic $C_{a,b}$ curves~\cite{harasawa2000fast}, specialized subsets of cases
such as superelliptic curves~\cite{shaska_2019}, or specific cases for example
$a = 3, b = 4$ have been worked on. Explicit
formulas for the case  $a=3$, $b=4$ have recently been developed
in~\cite{evan_g3}. The general pattern of composing and reducing divisor class
representatives to a reduced form is common among all algorithms mentioned. An
adaptation of NUCOMP to these settings may result in similar improvements seen
in the hyperelliptic curve setting.

Automating the generation of near-optimal explicit formulas for any genus based
on this work would also be of interest. This would provide further improvements for
$g > 3$ where developing formulas by hand is far too tedious, yielding
algorithms and implementations that are greatly improved over polynomial
arithmetic based algorithms. Choosing the right explicit techniques to apply
generically may prove to be challenging, because finding optimal explicit formulas
most certainly would require cycles of testing and optimization. The program
could apply as many combinations of techniques as possible and output explicit
formulas based on the intended application through specification inputs such as
the relative cost of field additions compared to field multiplications. This
type of work would also benefit from automatically generating testing scripts and
example inputs, that could be based on the testing scripts of this work.

At this point, divisor class arithmetic in genus 2 and 3 is quite thoroughly
optimized. Using ideas of the thesis (NUCOMP), there is still lots of room for
improvement in genus 4 and above, and perhaps for non-hyperelliptic curves.
Automation is possibly within reach as well for those cases. As discussed in
Section~\ref{sec:practicalConsiderations}, it is unlikely that arithmetic on
inert models will be needed in the future for numerical investigations, but if
that changes it should be possible to extend the ideas in this thesis to develop
efficient formulas for inert models as well. Perhaps the ideas of this work can
be applied to special families of curves, such as those with Kummer
surfaces~\cite{Gaudry_kummer_2007} and higher-genus analogues of DIK
curves~\cite{Doche_DIK_2006} or Edwards curves~\cite{edwards2007normal}. Isogeny
based cryptographic protocols (if divisor class arithmetic plays a role) could
be an application, or even direct improvements to isogeny computations
themselves. The impactful results of this thesis leave us with many interesting
avenues to explore.